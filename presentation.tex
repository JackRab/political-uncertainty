%\documentclass{beamer} 
\documentclass[handout]{beamer} 
\usetheme{Ilmenau}
\usepackage{graphicx,verbatim,hyperref}
\usepackage{textpos}

\usecolortheme{beaver}
\useinnertheme{default}
\setbeamertemplate{itemize item}[triangle]
\setbeamertemplate{itemize subitem}[triangle]
\setbeamertemplate{itemize subsubitem}[circle]
\setbeamertemplate{enumerate items}[default]
\setbeamertemplate{blocks}[upper=block head,rounded]
\setbeamercolor{item}{fg=black}
\usefonttheme{serif} %should allow ccfonts to take effect

\usepackage{cite}
\usepackage{times, verbatim,xcolor,bm}
%\usepackage[usenames,dvipsnames]{color}
\usepackage{amsbsy,amssymb, amsmath, amsthm}
\usepackage{booktabs}
%David miller's fonts
	\usepackage[T1]{fontenc}
	\usepackage[boldsans]{ccfonts}
	%\usepackage[boldsans]{concmath}
	\usepackage[euler-hat-accent]{eulervm}

\newcommand{\al}{\alpha}
\newcommand{\expect}{\mathbb{E}}
\newcommand{\Bt}{B(\bm{\tau^a})}
\newcommand{\bta}{\bm{\tau^a}}
\newcommand{\btn}{\bm{\tau^{tw}}}
\newcommand{\ga}{\gamma}
\newcommand{\ve}{\varepsilon}
\newcommand{\ta}{\theta}
\newcommand{\de}{\delta}
\newcommand{\ov}{\overline}
\newcommand{\un}{\underline}

\newenvironment{changemargin}[2]{% 
  \begin{list}{}{% 
    \setlength{\topsep}{0pt}% 
    \setlength{\leftmargin}{#1}% 
    \setlength{\rightmargin}{#2}% 
    \setlength{\listparindent}{\parindent}% 
    \setlength{\itemindent}{\parindent}% 
    \setlength{\parsep}{\parskip}% 
  }% 
  \item[]}{\end{list}} 
	
	\let\Tiny=\tiny


\title[Lobbying and Legislative Uncertainty\hspace{2.95in}\insertframenumber/\inserttotalframenumber]{Lobbying and Legislative Uncertainty}
%\author[Kristy Buzard]{Kristy Buzard \\ Syracuse University and The Wallis Institute \\ kbuzard@syr.edu}
\author[Kristy Buzard]{\texorpdfstring{Kristy Buzard\newline Syracuse University and The Wallis Institute  \newline\url{kbuzard@syr.edu}}{Kristy Buzard}}
\date{April 16, 2016}
\begin{document}
\maketitle
%\insertpresentationendpage removed b/c of appendix




\section{Overview}
\subsection{Preview}
\begin{frame}{The Questions}

\pause
\begin{enumerate}[<+->]
\item When 
	\begin{itemize}
		\item E
	\end{itemize}
	\vskip.1in
\item Can
	\begin{itemize}
		\item G
	\end{itemize}
\end{enumerate}

\end{frame}


\begin{frame}{Some Data}

\pause
\begin{enumerate}[<+->]
\item When 
	\begin{itemize}
		\item E
	\end{itemize}
	\vskip.1in
\item Can
	\begin{itemize}
		\item G
	\end{itemize}
\end{enumerate}

\end{frame}


\begin{frame}{Preview}
\frametitle{P}
\pause
With . I:
\pause
\begin{itemize}[<+->]
	\item e
	\item c
		\begin{itemize}[<+->]
			\item b
			\item t
		\end{itemize}
	\item e
\end{itemize}
\end{frame}


\begin{frame}{Preview}
\frametitle{Results}

\pause
\begin{itemize}[<+->]
	\item S
	\item F
	\item D
		\begin{itemize}
			\item P
			\item M
		\end{itemize}
\end{itemize}
\end{frame}

 
\begin{frame}{Literature}
\pause
\begin{itemize}[<+->]
	\item \textbf{Vote Buying in Legislatures}: 
	\item \textbf{Stochastic Voting}: 
	\item \textbf{Uncertainty}: 
\end{itemize}
\end{frame} 





\section{Model}
\subsection{Political Structure}
\begin{frame}{Policy and Politics}

\pause
T
\pause
\begin{itemize}
	\item A
	\pause
	\item B
\end{itemize}


\end{frame}



\begin{frame}{Timeline}
\pause
\begin{enumerate}[<+->]
	\item {\bfseries Who}
		\begin{enumerate}[i.]
			%\pause
			\item Governments 
		\end{enumerate}
	%\pause
	\item \textbf{Then}
		\begin{enumerate}[i.]
			%\pause
			\item A
			%\pause
			\item B
		\end{enumerate}
	%\pause
	\item \textbf{What}
	%\pause
		\begin{enumerate}[i.]
			\item If
		\end{enumerate}
\end{enumerate}
\end{frame}


\subsection{The Players}

\begin{frame}{D}
\pause
  Objective function:
\pause
\[
  W = \mathit{CS}_X(\tau) + \ga(s,e) \pi_X(\tau) + \mathit{CS}_Y(\tau^*) + \pi_Y(\tau^*) + \mathit{TR}(\tau)
\]
\vskip-.1in
\pause
\begin{itemize}[<+->]
	\item S
		\begin{itemize}
			\item $s$: 
			\item $e$: 
		\end{itemize}
	\item Optimal 
		\begin{itemize}
			\item Ignores 
			\item Takes 
		\end{itemize}
\end{itemize}
\end{frame}

\begin{frame}
\frametitle{Political Pressure}
Two potential sources
\pause
\begin{enumerate}[<+->]
	\item Endogenous effort choice of lobby, $e$
		\begin{itemize}[<+->]
			\item Lobby chooses effort to maximize profits, $\pi(\cdot)$, net of lobbying effort, $e$
			\item Call lobby's optimal effort choice $e^L$
						\[
						  e^L = \max_e \pi(\tau(\ga(e))) - e
						\]
		\end{itemize}
\end{enumerate}

\end{frame}




\section{Results}
\subsection{One Vote Buyer}
\begin{frame}{}

\pause
When :

\pause
\begin{itemize}[<+->]
	\item T
	\item I
	\item C
\end{itemize}

\pause
\vskip.2in
\begin{beamerboxesrounded}[upper=palette tertiary, shadow=true]{Result...}
    When Vote Buyer $B$ pays bribes to exactly two legislators, the bribes are such that the two bribed legislators' ideal points gross of bribes are equalized. Which two legislators are bribed depends on the bias parameter $\al$.
\end{beamerboxesrounded}

\end{frame}


\begin{frame}{When ...}
Now 
\pause
\begin{itemize}[<+->]
	\item Want 
	\item But 
\end{itemize}

\pause
\vskip.2in
\begin{beamerboxesrounded}[upper=palette tertiary, shadow=true]{Result...}
    When Vote Buyer $B$ pays bribes to all three legislators, the bribes are such that the legislators' ideal points gross of bribes are equalized.
\end{beamerboxesrounded}
\end{frame}

\begin{frame}
\begin{beamerboxesrounded}[upper=palette tertiary, shadow=true]{Result...}
      When Vote Buyer $B$ pays bribes to exactly one legislator, it may be any one of the three legislators depending on the bias parameter $\al$.
\end{beamerboxesrounded}

\pause
\vskip.2in
\begin{beamerboxesrounded}[upper=palette tertiary, shadow=true]{Result...}
  When Vote Buyer $B$ has a low willingness to pay, he does not bribe any legislator.
\end{beamerboxesrounded}

\end{frame}


\begin{frame}{Varying Uncertainty Across Legislators}
Now 
\pause
\begin{itemize}[<+->]
	\item Want 
	\item But 
\end{itemize}

\pause
\vskip.2in
\begin{beamerboxesrounded}[upper=palette tertiary, shadow=true]{Conjecture}
 When there is no bias in the positions of the legislators ($\al =0$), the bribes of legislators whose ideal points are at the median in terms of uncertainty receive the highest relative bribes.\end{beamerboxesrounded}
\end{frame}

\subsection{Two Vote Buyers}
\begin{frame}
\begin{beamerboxesrounded}[upper=palette tertiary, shadow=true]{Result...}
  It is possible that neither vote buyer bribes any legislator on a given vote. This occurs when both vote buyers' willingness-to-pay parameters are small.
\end{beamerboxesrounded}

\pause
\vskip.2in
\begin{beamerboxesrounded}[upper=palette tertiary, shadow=true]{Result...}
  It is possible for both vote buyers to bribe legislators on the same vote.
\end{beamerboxesrounded}

\end{frame}



\section{Conclusion}

\begin{frame}{Future Work}
\pause
\begin{itemize}[<+->]
	\item A
	\item I
	\item C
\end{itemize}

\end{frame}


\begin{frame}{Conclusion}
Taking into account
\pause
\begin{itemize}[<+->]
		\item provides 
		\item demonstrates 
		\item helps 
\end{itemize}

\end{frame}


\end{document}